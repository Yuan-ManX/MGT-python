\documentclass[9pt]{extarticle}
\usepackage[utf8]{inputenc}
\usepackage[T1]{fontenc}
\usepackage{float}
\usepackage{titling}
\usepackage{geometry}
\usepackage{authblk}
\usepackage{enumitem}

\title{Musical Gestures Toolbox\\ \large{Documentation}}
\author{Frida Furmyr \& Marcus Widmer}

\begin{document}
\maketitle

\tableofcontents

\section{Class MgObject}
    Initializes Musical Gestures data structure from a given parameter video file.\\

    \noindent Parameters:
    \begin{itemize}
   \item [] filename (str): Name of input parameter video file.
   \item [] method (str): Currently 'Diff' is the only implemented method. 
   \item [] filtertype (str): 'Regular', 'Binary', 'Blob' (see function motionfilter).
    \item [] thresh (float): a number in [0,1]. Eliminates pixel values less than given threshold.
   \item [] starttime (float): cut the video from this start time (min) to analyze what is relevant.
   \item [] endtime (float): cut the video at this end time (min) to analyze what is relevant.
   \item []  blur (str): 'Average' to apply a blurring filter, 'None' otherwise.
   \item[] skip (int): When proceeding to analyze next frame of video, this many frames are skipped. NB: skip cannot exceed number of frames per second (fps). 
    \item [] color (bool): True does the analysis in RGB, False in grayscale. 
    \item [] contrast (float): apply +/- 100 contrast to video
   \item []  brightness (float): apply +/- 100 brightness to video
    \item [] crop (str): 'none', 'manual', 'auto' to select cropping of relevant video frame size
    \end{itemize}
\section{MgObject methods}

\subsection{mg\_videoreader}
    mg\_videoreader(filename, starttime = 0, endtime = 0, skip = 0, contrast = 0, brightness = 0, crop = 'none'):
    \\\\
    Reads in a video file, and by input parameters user decide if it: trims the length, skips frames, applies contrast/brightness adjustments and/or crops image width/height.
    \\\\
    Parameters:
    \begin{itemize}
     \item [] filename (str): Name of input parameter video file.
     \item [] starttime (float): cut the video from this start time (min) to analyze what is relevant.
     \item [] endtime (float): cut the video at this end time (min) to analyze what is relevant.
     \item [] skip (int): When proceeding to analyze next frame of video, this many frames are skipped.
     \item [] contrast (float): apply +/- 100 contrast to video
     \item [] brightness (float): apply +/- 100 brightness to video
     \item [] crop (str): 'None', 'Auto' or 'Manual' to crop video.
    \end{itemize}
    Returns:
    \begin{itemize}
    \item [] vidcap (VideoCapture object): cv2 video capture of editevideo file
    \item [] length (int), fps(int), width(int), height(int): video attributes
     \item [] of (str): only-filename - filename gets updated with what procedures it went through.
 \end{itemize}

\subsection{mg\_motionvideo}    
    mg\_motionvideo(self, method = 'Diff', filtertype = 'Regular', thresh = 0.001, blur = 'None', kernel\_size = 5, inverted\_motionvideo = False, inverted\_motiongram = True, unit = 'seconds', enhance = 1):
    \\\\
    Finds the difference in pixel value from one frame to the next in an input video, and saves the frames into a new video.
    Describes the motion in the recording.    
    Outputs a video called filename +'\_motion.avi'.
    \\\\
    Parameters:
    
    \begin{itemize}
    \item [] kernel\_size (int): Size of structuring element.
   \item [] method (str): Currently 'Diff' is the only implemented method. 
    \item [] filtertype (str): 'Regular', 'Binary', 'Blob'(see function motionfilter) 
    \item [] thresh (float): a number in [0,1]. Eliminate spixel values less than given threshold.
    \item [] blur (str): 'Average' to apply a blurring filter, 'None' otherwise.
    \item [] inverted\_motionvideo (bool): Inverts colors of motionvideo
    \item [] inverted\_motiongram (bool): Inverts colors of motiongram
    \item [] unit (str): Unit in QoM plot. 'seconds' or 'samples'
    \item [] equalize\_motiongram (bool): Converts the motiongram to hsv-color space and flattens the value channel (v).
    \end{itemize}
    Returns:
    \begin{itemize}
    \item [] None
    \end{itemize}


\subsection{mg\_cropvideo}

    mg\_cropvideo(fps,width,height, length, of, crop\_movement = 'auto', motion\_box\_thresh = 0.1, motion\_box\_margin = 1)
    \\\\
	Crops the video.\\\\
Parameters:
        \begin{itemize}
		\item [] crop\_movement(str): 'auto' finds the bounding box that contains the total motion in the video. 'manual' opens up a simple GUI that is used to crop the video manually by looking at the first frame.

		\item [] motion\_box\_thresh (float): Only meaningful is crop\_movement = 'auto'. Takes floats between 0 and 1, where 0 includes all the motion and 1 includes none.
		
		\item [] motion\_box\_margin (int): Only meaningful is crop\_movement = 'auto'. Add margin to the bounding box.
        \end{itemize}
	Returns:
    \begin{itemize}
	   \item [] None
    \end{itemize}


\subsection{mg\_input\_test}
    mg\_input\_test(filename,method,filtertype,thresh,starttime,endtime,blur,skip): 
    \\\\
    Gives feedback to user if initialization from input went wrong.
    Ex: raise InputError(msg)
    msg = 'Please specify a filter type as str: Regular or Binary'


\subsection{mg\_motionhistory} 
    mg\_motionhistory(self, history\_length = 20, kernel\_size = 5, method = 'Diff', filtertype = 'Regular', thresh = 0.001, blur = 'None', inverted\_motionhistory = False):
    \\\\
    Finds the difference in pixel value from one frame to the next in an input video, and saves the difference frame to a history tail. 
    The history frames are summed up and normalized, and added to the current difference frame to show the history of motion. \\
    Outputs a video called filename + '\_motionhistory.avi'.
\\\\
    Parameters:
    \begin{itemize}
    \item []history\_length (int): How many frames will be saved to the history tail.
    \item []kernel\_size (int): Size of structuring element.
    \item []method (str): Currently 'Diff' is the only implemented method. 
    \item []filtertype (str): 'Regular', 'Binary', 'Blob' (see function motionfilter) 
	\item []thresh (float): a number in [0,1]. Eliminates pixel values less than given threshold.
    \item []blur (str): 'Average' to apply a blurring filter, 'None' otherwise.
    \item [] inverted\_motionhistory (bool): Inverts the colors of motionhistory video
    \end{itemize}
    Returns:
    \begin{itemize}
    \item [] None
\end{itemize}

\subsection{mg\_contrast\_brightness}
    mg\_contrast\_brightness(of,vidcap,fps,width,height,contrast,brightness):\\\\
    Edit contrast and brightness of the video.\\\\
    Parameters:
    \begin{itemize}
    \item []of (str): filename without extension
    \item [] vidcap (VideoCapture object): cv2 capture of video file, with all frames ready to read with vidcap.read().
    \item [] fps (int), width (int), height (int) are simply info about vidcap
    \item [] contrast (float): apply +/- 100 contrast to video
    \item [] brightness (float): apply +/- 100 brightness to video
    \end{itemize}
    Returns:
    \begin{itemize}
    \item [] vidcap (VideoCapture object): cv2 video capture of edited video file
    \end{itemize}


\subsection{mg\_skip\_frames}
    mg\_skip\_frames(of, vidcap, skip, fps, width, height):
    \\\\
    Frame skip, convenient for saving time/space in an analysis of less detail looking at big picture movement. Skips the given number of frames, making a compressed version of the input video file.
    \\\\
    Parameters:
    \begin{itemize}
    \item []of (str): filename without extension
    \item []vidcap (VideoCapture object): cv2 capture of video file, with all frames ready to read with vidcap.read().
    \item []fps (int), width (int), height (int) are simply info about vidcap
    \item []skip (int): When proceeding to analyze next frame of video, this many frames are skipped.
    \end{itemize}
    Returns:
    \begin{itemize}
       \item [] vidcap (VideoCapture object): cv2 video capture of edited video file.
       \item [] length (int), fps (int), width (int), height(int): new video attributes.
   \end{itemize}


\subsection{mg\_show}
mg\_show(self, filename = None):
\\\\
This function simply plays the current vidcap VideoObject. The speed of the video playback 
might not match the true fps due to non-optimized code. 
\\\\
Parameters:
\begin{itemize}
\item[] filename(str): If left empty, the current vidcap object is played. If filename is given,
this file is played instead.
\end{itemize}
Returns:
\begin{itemize}
    \item[] None
\end{itemize}



\section{Functions}
\subsection{average\_image}
    average\_image(filename, enhance = 0):
    \\\\
    Post-processing tool. Finds and saves an average image of entire video. \\\\Usage:\\
    from \_motionaverage import motionaverage
    \\motionaverage('filename.avi', enhance = 0.5)
    \\\\
    Parameters:
    \begin{itemize}
        \item [] filename (str): name of video 
        \item [] enhance (float): takes values between '0' and '1', where '0' is no enhancement and '1' scales the pixel values such that the brightest pixel gets the value 255. 
    \end{itemize}
    Returns:
    \begin{itemize}
        \item [] None
    \end{itemize}

\subsection{history}
history(filename,history\_length = 10):\\\\
This function  creates a video where each frame is the average of the $n$ previous frames, where $n$ is determined
from history\_length.
The history frames are summed up and normalized, and added to the current frame to show the history. \\
Outputs a video called filename + '\_history.avi'.
\\\\
Parameters:
\begin{itemize}
\item[] history\_length (int): How many frames will be saved to the history tail.
\end{itemize}
Returns:
\begin{itemize}
\item[] None
\end{itemize}

\subsection{centroid}
    centroid(image, width, height):
    \\\\
    Computes the centroid of motion and quantity of motion of an image/frame.
    \\\\
    Parameters:
    \begin{itemize}
    \item [] image (numpy array(uint8))
    \item [] width (int)
    \item [] height (int)
    \end{itemize}
    Returns:
    \begin{itemize}
    \item [] Centroid of motion ([X(int),Y(int)]): X,Y(origo in bottom left corner) coordinate of the maximum change in pixel value
    \item [] Quantity of motion (int): How large was the change in pixel value
    \end{itemize}

\subsection{constrainNumber}
    constrainNumber(n, minn, maxn)\\\\
    Constrains number to having a value between minn and maxn
    \\\\
    Parameters:
    \begin{itemize}
    \item [] n (int, float): number to constrain
    \item [] minn (int, float): lower limit n can be
    \item [] maxn (int, float): lower limit n can be
    \end{itemize}
    Returns:
    \begin{itemize}
    \item [] Constrained number(int, float)
\end{itemize}

\subsection{filter\_frame}
    filter\_frame(motion\_frame, filtertype,thresh,kernel\_size)
    \\\\
    Apply a filter to a picture/videoframe
    \\\\
    Parameters:
    \begin{itemize}
    \item [] motion\_frame (array(uint8)): input motion image
    \item [] filtertype (str): ’Regular’, turns all values below thresh to 0, ’Binary’ turns all values below thresh to 0, above thresh to 1, ’Blob’ removes individual pixels with erosion method.
    \item [] thresh (float): for ’Regular’ and ’Binary’ option, thresh is a value of threshold [0,1];
    \item [] kernel\_size(int): Size of structuring element
    \end{itemize}
    Returns: 
    \begin{itemize}
    \item [] filtered frame (array(uint8))
    \end{itemize}

\end{document}