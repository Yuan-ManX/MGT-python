\documentclass[9pt]{extarticle}
\usepackage[utf8]{inputenc}
\usepackage[T1]{fontenc}
\usepackage{float}
\usepackage{titling}
\usepackage{geometry}
\usepackage{authblk}
\usepackage{enumitem}

\def\code#1{\texttt{#1}}

\title{Musical Gestures Toolbox\\ \large{Documentation}}
\author{Balint Laczko}

\begin{document}
\maketitle

\tableofcontents


\section{Class MgObject}
Initializes Musical Gestures data structure from a video file.\\

\noindent Attributes

\begin{itemize}

\item filename : str\\
Path to the video file.

\item filtertype : \{'Regular', 'Binary', 'Blob'\}, optional\\
The \code{filtertype} parameter for the \code{motion()} method. 
\code{Regular} turns all values below \code{thresh} to 0. 
\code{Binary} turns all values below \code{thresh} to 0, 
above \code{thresh} to 1. \code{Blob} removes individual pixels 
with erosion method.

\item thresh : float, optional\\
The \code{thresh} parameter for the \code{motion()} method. 
A number in the range of 0 to 1. Default is 0.05. Eliminates pixel 
values less than given threshold.

\item starttime : int or float, optional\\
Trims the video from this start time (s).

\item endtime : int or float, optional\\
Trims the video until this end time (s).

\item blur : \{'None', 'Average'\}, optional\\
The \code{blur} parameter for the \code{motion()} method.
\code{Average} to apply a 10px * 10px blurring filter, \code{None} 
otherwise.

\item skip : int, optional\\
Time-shrinks the video by skipping (discarding) 
every n frames determined by \code{skip}.

\item rotate : int or float, optional\\
Rotates the video by a \code{rotate} degrees.

\item color : bool, optional\\
Default is \code{True}. If \code{False}, converts the video to grayscale 
and sets every method in grayscale mode.

\item contrast : int or float, optional\\
Applies +/- 100 contrast to video.

\item brightness : int or float, optional\\
Applies +/- 100 brightness to video.

\item crop : \{'none', 'manual', 'auto'\}, optional\\
If \code{manual}, opens a window displaying the first frame of the 
input video file, where the user can draw a rectangle to which 
cropping is applied. If \code{auto} the cropping function attempts to 
determine the area of significant motion and applies the cropping 
to that area.

\item keep\_all : bool, optional\\
Default is \code{False}. If \code{True}, preserves an output video file 
after each used preprocessing stage.

\end{itemize}


\section{MgObject methods}


\subsection{show}

\code{show(
    self, 
    filename=None, 
    key=None):}
\\\\
This function simply plays the current vidcap VideoObject. 
The speed of the video playback might not match the true fps 
due to non-optimized code. 
\\\\

\noindent Parameters
\begin{itemize}
\item filename : str, optional

Default is \code{None}. If \code{None}, the current video to which 
the MgObject points is played. If filename is given, this file is 
played instead. 
\item key : \{None, 'mgx', 'mgy', 'average', 'plot', 'motion', 'history', 'motionhistory', 'sparse', 'dense'\}, optional

If either of these shorthands is used the method attempts to show the 
(previously rendered) video file corresponding to the one in the MgObject.
\end{itemize}


\subsection{motion}

\code{motion(
    self, 
    filtertype='Regular', 
    thresh=0.05, 
    blur='None',
    kernel\_size=5, 
    inverted\_motionvideo=False, 
    inverted\_motiongram=False,
    unit='seconds', 
    equalize\_motiongram=True, 
    save\_plot=True, 
    save\_data=True, 
    data\_format="csv", 
    save\_motiongrams=True, 
    save\_video=True):}
\\\\
Finds the difference in pixel value from one frame to the next in an 
input video, and saves the frames into a new video. Describes the 
motion in the recording.
\\\\

\noindent Parameters

\begin{itemize}

\item filtertype : \{'Regular', 'Binary', 'Blob'\}, optional

\code{Regular} turns all values below \code{thresh} to 0.
\code{Binary} turns all values below \code{thresh} to 0, 
above \code{thresh} to 1. \code{Blob} removes individual pixels 
with erosion method.

\item thresh : float, optional

A number in the range of 0 to 1. Default is 0.05.
Eliminates pixel values less than given threshold.

\item blur : \{'None', 'Average'\}, optional

\code{Average} to apply a 10px * 10px blurring filter, \code{None}
otherwise.

\item kernel\_size : int, optional

Default is 5. Size of structuring element.

\item inverted\_motionvideo : bool, optional

Default is \code{False}. If \code{True}, inverts colors 
of the motion video.

\item inverted\_motiongram : bool, optional

Default is \code{False}. If \code{True}, inverts colors 
of the motiongrams.

\item unit : \{'seconds', 'samples'\}, optional

Unit in QoM plot.

\item equalize\_motiongram : bool, optional

Default is \code{True}. If \code{True}, converts the 
motiongrams to hsv-color space and flattens the value 
channel (v).

\item save\_plot : bool, optional

Default is \code{True}. If \code{True}, outputs motion-plot.

\item save\_data : bool, optional

Default is \code{True}. If \code{True}, outputs motion-data.

\item data\_format : \{'csv', 'tsv', 'txt'\}, optional

Specifies format of motion-data.

\item save\_motiongrams : bool, optional

Default is \code{True}. If \code{True}, outputs motiongrams.

\item save\_video : bool, optional

Default is \code{True}. If \code{True}, outputs the motion video.

\end{itemize}


\noindent Outputs

\begin{itemize}

\item \code{filename}\_motion.avi

A video of the absolute difference between consecutive frames 
in the source video. 

\item \code{filename}\_motion\_com\_qom.png

A plot describing the centroid of motion and the quantity of 
motion in the source video.

\item \code{filename}\_mgx.png

A horizontal motiongram of the source video.

\item \code{filename}\_mgy.png

A vertical motiongram of the source video.

\item \code{filename}\_motion.csv

A text file containing the quantity of motion and the centroid 
of motion for each frame in the source video with timecodes in 
milliseconds. Available formats: csv, tsv, txt.

\end{itemize}

\noindent Returns

\begin{itemize}
\item MgObject 

A new MgObject pointing to the output '\_motion' video file. 
If \code{save\_video=False}, it returns an MgObject 
pointing to the input video file.
\end{itemize}


\subsection{history}

\code{history(
    self, 
    filename='', 
    history\_length=10):}
\\\\
This function  creates a video where each frame is the average of the 
n previous frames, where n is determined by \code{history\_length}.
The history frames are summed up and normalized, and added to the 
current frame to show the history.
\\\\

\noindent Parameters
\begin{itemize}
\item filename : str, optional

Path to the input video file. If not specified the video file 
pointed to by the MgObject is used.

\item history\_length : int, optional

Default is 10. Number of frames to be saved in the history tail.
\end{itemize}

\noindent Outputs
\begin{itemize}
\item \code{filename}\_history.avi
\end{itemize}

\noindent Returns
\begin{itemize}
\item MgObject 
A new MgObject pointing to the output '\_history' video file.
\end{itemize}


\subsection{motionhistory}

\code{motionhistory(
    self,
    history\_length=10,
    kernel\_size=5,
    filtertype='Regular',
    thresh=0.05,
    blur='None',
    inverted\_motionhistory=False):}
\\\\
Finds the difference in pixel value from one frame to the next in an input video, 
and saves the difference frame to a history tail. The history frames are summed up 
and normalized, and added to the current difference frame to show the history of 
motion. 
\\\\

\noindent Parameters
\begin{itemize}
\item history\_length : int, optional

Default is 10. Number of frames to be saved in the history tail.

\item kernel\_size : int, optional

Default is 5. Size of structuring element.

\item filtertype : \{'Regular', 'Binary', 'Blob'\}, optional

\code{Regular} turns all values below \code{thresh} to 0.
\code{Binary} turns all values below \code{thresh} to 0, 
above \code{thresh} to 1. \code{Blob} removes individual pixels 
with erosion method.

\item thresh : float, optional

A number in the range of 0 to 1. Default is 0.05.
Eliminates pixel values less than given threshold.

\item blur : \{'None', 'Average'\}, optional

\code{Average} to apply a 10px * 10px blurring filter, 
\code{None} otherwise.

\item inverted\_motionhistory : bool, optional

Default is \code{False}. If \code{True}, inverts colors 
of the motionhistory video.
\end{itemize}

\noindent Outputs
\begin{itemize}
\item \code{filename}\_motionhistory.avi
\end{itemize}

\noindent Returns
\begin{itemize}
\item MgObject

A new MgObject pointing to the output '\_motionhistory' video file.
\end{itemize}


\subsection{average}

\code{average(
    self, 
    filename='', 
    normalize=True):}
\\\\
Finds and saves an average image of an input video file.
\\\\

\noindent Parameters
\begin{itemize}
\item filename : str, optional

Path to the input video file. If not specified the video 
file pointed to by the MgObject is used.

\item normalize : bool, optional

Default is \code{True}. If \code{True}, normalizes pixel values in 
the output image.

\end{itemize}

\noindent Outputs
\begin{itemize}
\item \code{filename}\_average.png
\end{itemize}

\noindent Returns
\begin{itemize}
\item MgImage

A new MgImage pointing to the output '\_average' image file.
\end{itemize}


\subsection{flow.sparse}

\code{sparse(
    self,
    filename='',
    corner\_max\_corners=100,
    corner\_quality\_level=0.3,
    corner\_min\_distance=7,
    corner\_block\_size=7,
    of\_win\_size=(15, 15),
    of\_max\_level=2,
    of\_criteria=(cv2.TERM\_CRITERIA\_EPS | cv2.TERM\_CRITERIA\_COUNT, 10, 0.03)):}
\\\\
Renders a sparse optical flow video of the input video file 
using \code{cv2.calcOpticalFlowPyrLK()}. \code{cv2.goodFeaturesToTrack()} 
is used for the corner estimation. For more details about the parameters 
consult the cv2 documentation.
\\\\

\noindent Parameters
\begin{itemize}
\item filename : str, optional

Path to the input video file. If not specified the video 
file pointed to by the MgObject is used.

\item corner\_max\_corners : int, optional

Default is 100.

\item corner\_quality\_level : float, optional

Default is 0.3.

\item corner\_min\_distance : int, optional

Default is 7.

\item corner\_block\_size : int, optional

Default is 7.

\item of\_win\_size : tuple (int, int), optional

Default is (15, 15).

\item of\_max\_level : int, optional

Default is 2.

\item of\_criteria : optional

Default is \code{(cv2.TERM\_CRITERIA\_EPS | cv2.TERM\_CRITERIA\_COUNT, 10, 0.03)}.
\end{itemize}

\noindent Outputs
\begin{itemize}
\item \code{filename}\_flow\_sparse.avi
\end{itemize}

\noindent Returns
\begin{itemize}
\item MgObject

A new MgObject pointing to the output '\_flow\_sparse' video file.
\end{itemize}


\subsection{flow.dense}

\code{dense(
    self,
    filename='',
    pyr\_scale=0.5,
    levels=3,
    winsize=15,
    iterations=3,
    poly\_n=5,
    poly\_sigma=1.2,
    flags=0,
    skip\_empty=False):}
\\\\
Renders a dense optical flow video of the input video file 
using \code{cv2.calcOpticalFlowFarneback()}.
For more details about the parameters consult the cv2 documentation.
\\\\

\noindent Parameters
\begin{itemize}
\item filename : str, optional

Path to the input video file. If not specified the video file 
pointed to by the MgObject is used.

\item pyr\_scale : float, optional

Default is 0.5.

\item levels : int, optional

Default is 3.

\item winsize : int, optional

Default is 15.

\item iterations : int, optional

Default is 3.

\item poly\_n : int, optional

Default is 5.

\item poly\_sigma : float, optional

Default is 1.2.

\item flags : int, optional

Default is 0.

\item skip\_empty : bool, optional

Default is \code{False}. If \code{True}, repeats previous frame 
in the output when encounters an empty frame.
\end{itemize}

\noindent Outputs
\begin{itemize}
\item \code{filename}\_flow\_dense.avi
\end{itemize}

\noindent Returns
\begin{itemize}
\item MgObject

A new MgObject pointing to the output '\_flow\_dense' video file.
\end{itemize}



\section{Utility Functions}


\subsection{centroid}

\code{centroid(
    image, 
    width, 
    height):}
\\\\
Defined in \textit{\_centroid.py}.\\
Computes the centroid of an image or frame.
\\\\
\noindent Parameters
\begin{itemize}
\item image : np.array(uint8)

The input image matrix for the centroid estimation function. 

\item width : int

The pixel width of the input video capture. 

\item height : int

The pixel height of the input video capture. 
\end{itemize}

\noindent Returns
\begin{itemize}
\item np.array(2)

X and Y coordinates of the centroid of motion.

\item int

Quantity of motion: How large the change was in pixels.
\end{itemize}


\subsection{mg\_cropvideo}

\code{mg\_cropvideo(
    fps,
    width,
    height,
    length,
    of,
    fex,
    crop\_movement='Auto',
    motion\_box\_thresh=0.1,
    motion\_box\_margin=1):}
\\\\
Defined in \textit{\_cropvideo.py}.\\
Crops the video.
\\\\

\noindent Parameters
\begin{itemize}
\item fps : int

The FPS (frames per second) of the input video capture.

\item width : int

The pixel width of the input video capture. 

\item height : int

The pixel height of the input video capture. 

\item length : int

The number of frames in the input video capture.

\item of : str

'Only filename' without extension (but with path to the file).

\item fex : str

File extension.

\item crop\_movement : \{'Auto','Manual'\}, optional

\code{Auto} finds the bounding box that contains the total motion in 
the video. Motion threshold is given by motion\_box\_thresh. \code{Manual} 
opens up a simple GUI that is used to crop the video manually by 
looking at the first frame.

\item motion\_box\_thresh : float, optional

Only meaningful if \code{crop\_movement='Auto'}. Takes floats between 
0 and 1, where 0 includes all the motion and 1 includes none.

\item motion\_box\_margin : int, optional

Only meaningful if \code{crop\_movement='Auto'}. Adds margin to 
the bounding box.
\end{itemize}


\subsection{filter\_frame}

\code{filter\_frame(
    motion\_frame, 
    filtertype, 
    thresh, 
    kernel\_size):}
\\\\
Defined in \textit{\_filter.py}.\\
Applies a filter to an image or videoframe.
\\\\

\noindent Parameters
\begin{itemize}
\item motion\_frame : np.array(uint8) 

    Input motion image.
\item filtertype : \{'Regular', 'Binary', 'Blob'\}

\code{Regular} turns all values below \code{thresh} to 0.
\code{Binary} turns all values below \code{thresh} to 0, 
above \code{thresh} to 1. \code{Blob} removes individual pixels 
with erosion method.

\item thresh : float

A number in the range of 0 to 1.
Eliminates pixel values less than given threshold.

\item kernel\_size : int

Size of structuring element.
\end{itemize}

\noindent Returns
\begin{itemize}
\item np.array(uint8)

Filtered frame.
\end{itemize}


\subsection{mg\_contrast\_brightness}

\code{mg\_contrast\_brightness(
    of, 
    fex, 
    vidcap, 
    fps, 
    length, 
    width, 
    height, 
    contrast, 
    brightness):}
\\\\
Defined in \textit{\_videoadjust.py}.\\
Applies contrast and brightness to a video.
\\\\

\noindent Parameters
\begin{itemize}
\item of : str

'Only filename' without extension (but with path to the file).

\item fex : str

File extension.

\item vidcap : 

cv2 capture of video file, with all frames ready to be 
read with \code{vidcap.read()}.

\item fps : int

The FPS (frames per second) of the input video capture.

\item length : int

The number of frames in the input video capture.

\item width : int

The pixel width of the input video capture. 

\item height : int

The pixel height of the input video capture. 

\item contrast : int or float, optional

Applies +/- 100 contrast to video.

\item brightness : int or float, optional

Applies +/- 100 brightness to video.
\end{itemize}

\noindent Outputs
\begin{itemize}
\item A video file with the name \code{of} + 'cb' + \code{fex}.
\end{itemize}

\noindent Returns
\begin{itemize}
\item cv2 video capture of output video file.
\end{itemize}


\subsection{mg\_skip\_frames}

\code{mg\_skip\_frames(
    of, 
    fex, 
    vidcap, 
    skip, 
    fps, 
    length, 
    width, 
    height):}
\\\\
Defined in \textit{\_videoadjust.py}.\\
Time-shrinks the video by skipping (discarding) every n frames 
determined by \code{skip}.
\\\\

\noindent Parameters
\begin{itemize}
\item of : str

'Only filename' without extension (but with path to the file).

\item fex : str

File extension.

\item vidcap : 

cv2 capture of video file, with all frames ready to be 
read with \code{vidcap.read()}.
\item skip : int

Every n frames to discard. \code{skip=0} keeps all frames, 
\code{skip=1} skips every other frame.

\item fps : int

The FPS (frames per second) of the input video capture.

\item length : int

The number of frames in the input video capture.

\item width : int

The pixel width of the input video capture. 
\item height : int

The pixel height of the input video capture.
\end{itemize}

\noindent Outputs
\begin{itemize}
\item A video file with the name \code{of} + '\_skip' + \code{fex}.
\end{itemize}

\noindent Returns
\begin{itemize}
\item videcap :

cv2 video capture of output video file.

\item length : int

The number of frames in the output video file.

\item fps : int

The FPS (frames per second) of the output video file.

\item width : int

The pixel width of the output video file. 

\item height : int

The pixel height of the output video file. 
\end{itemize}



\subsection{mg\_videoreader}

\code{mg\_videoreader(
    filename,
    starttime=0,
    endtime=0,
    skip=0,
    rotate=0,
    contrast=0,
    brightness=0,
    crop='None',
    color=True,
    keep\_all=False,
    returned\_by\_process=False):}
\\\\
Defined in \textit{\_videoreader.py}.\\
Reads in a video file, and optionally apply several different 
processes on it. These include:
\begin{itemize}
\item trimming,
\item skipping,
\item rotating,
\item applying brightness and contrast,
\item cropping,
\item converting to grayscale.
\end{itemize}

\noindent Parameters
\begin{itemize}
\item filename : str

Path to the input video file.

\item starttime : int or float, optional

Trims the video from this start time (s).

\item endtime : int or float, optional

Trims the video until this end time (s).

\item skip : int, optional

Time-shrinks the video by skipping (discarding) every n 
frames determined by \code{skip}.

\item rotate : int or float, optional

Rotates the video by a \code{rotate} degrees.

\item contrast : int or float, optional

Applies +/- 100 contrast to video.

\item brightness : int or float, optional

Applies +/- 100 brightness to video.

\item crop : \{'none', 'manual', 'auto'\}, optional

If \code{manual}, opens a window displaying the first frame of the 
input video file, where the user can draw a rectangle to which 
cropping is applied. If \code{auto} the cropping function attempts to 
determine the area of significant motion and applies the cropping 
to that area.

\item color : bool, optional

Default is \code{True}. If \code{False}, converts the video to 
grayscale and sets every method in grayscale mode.

\item keep\_all : bool, optional

Default is \code{False}. If \code{True}, preserves an output 
video file after each used preprocessing stage.
\end{itemize}

\noindent Outputs
\begin{itemize}
\item A video file with the applied processes. The name of the file 
will be \code{filename} + a suffix for each process.
\end{itemize}

\indent Returns
\begin{itemize}
\item length : int

The number of frames in the output video file.

\item width : int

The pixel width of the output video file. 

\item height : int

The pixel height of the output video file. 

\item fps : int

The FPS (frames per second) of the output video file.

\item endtime : float

The length of the output video file in seconds.

\item of: str

The path to the output video file without its extension.
The file name gets a suffix for each used process.

\item fex : str

The file extension of the output video file.
Currently it is always 'avi'.
\end{itemize}











\end{document}